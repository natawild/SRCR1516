\documentclass[pdftex,12pt,a4paper]{report}

\usepackage[pdftex]{graphicx}
\usepackage{float}
\usepackage{fancyvrb}
\fvset{xleftmargin=2em}

\usepackage{pgfplots}
\pgfplotsset{width=10cm,compat=1.9}
\usepackage{tikzscale}
\usepackage{pgfplotstable}
\usepackage{booktabs}
\usepackage[font=small,labelfont=bf,tableposition=top]{caption}

\usepackage[utf8]{inputenc} % isto é um comentário
\usepackage[portuges]{babel}
\usepackage[T1]{fontenc}
\usepackage{times}
%\usepackage{lmodern}
\usepackage[obeyspaces,spaces]{url}
\usepackage[left=25mm,right=25mm,top=25mm,bottom=25mm]{geometry}
\usepackage{titlesec}
\usepackage{mathtools}
%identa 1º paragrafo de capitulos e secções
\usepackage{indentfirst}
\usepackage{enumerate}
\usepackage{listings}
\lstset{
	basicstyle=\small,
	numbers=left,
	numberstyle=\tiny,
	numbersep=5pt,
	breaklines=true,
	frame=tB,
	mathescape=true,
	escapeinside={(*@}{@*)}
}

\usepackage[titletoc,toc]{appendix}
\renewcommand{\appendixtocname}{Anexos}


\newcommand{\HRule}{\rule{\linewidth}{0.5mm}}
\titleformat{\chapter}{\normalfont\huge}{\thechapter.}{20pt}{\huge}


\begin{document}

\begin{titlepage}


\begin{minipage}{0.3\textwidth}
\begin{flushleft} 
\includegraphics[width=\textwidth]{./logo.png}
\end{flushleft}
\end{minipage}
\begin{minipage}{0.6\textwidth}
\begin{flushright} 

\textsc{Departamento de Engenharia Informática}\\[0.1cm]
\bfseries Mestrado Integrado em Engenharia Informática \\ [0.1cm]
\bfseries \textit{Sistemas de Representação Conhecimento e Racocínio}\\[8mm]

\end{flushright}
\end{minipage}


\vspace{3cm}


\begin{center}


\LARGE Exercício 1

\Large Registo de eventos numa instituição de saúde\\[1.5cm]


{\Large \bfseries Grupo 19\\[2cm] }

\begin{minipage}{0.4\textwidth}
		\large Célia Natália Lemos Figueiredo\\
           Aluna a67367
\end{minipage}
\\[1cm]
\begin{minipage}{0.4\textwidth}
		\large Gil Gonçalves \\
           Aluno a67738\\
\end{minipage}
\\[1cm]
\begin{minipage}{0.4\textwidth}
		\large José Carlos Pedrosa Lima de Faria\\
		Aluno  a67638
\end{minipage}
\\[1cm]
\begin{minipage}{0.4\textwidth}
	\large Judson Quissanga Coge Paiva\\
	Aluno  E6846
\end{minipage}


\vfill

\large Braga, {\large \today}

\end{center}
\end{titlepage}


\tableofcontents
\listoffigures 

\begin{abstract}

O presente relatório documenta o primeiro trabalho prático da Unidade Curricular de Sistemas de Representação Conhecimento e Racocínio.
Nesta primeira fase o objetivo foi construir um mecanismo de representação de conhecimento para o registo de eventos numa instituição de saúde. Em termos gerais, foi usada a linguagem PROLOG, esta que utiliza um conjunto de fatos, predicados e regras de derivação de lógica. Uma execução de um programa é, na verdade, uma prova de um teorema, iniciada por uma consulta. 
Neste relatório pretende-se apresentar a forma como a aplicação foi construída bem como explicar algumas decisões tomadas.




O relatório encontra-se organizado em 3 partes. Na primeira parte apresentam-se as classes usadas, explicando qual a função de cada uma e das suas respectivas funções. Na segunda parte apresenta-se o gerador. Indica-se quais as figuras para as quais é possível gerar pontos bem como de que forma os pontos de cada figura são gerados. Na última parte apresenta-se o motor, nomeadamente a forma como os ficheiros de pontos são lidos, como os pontos são desenhados, bem como algumas considerações sobre a câmara e menus da aplicação.


\end{abstract}
\chapter{Introdução}
\label{cap:p1}

Nesta secção apresentam-se as classes usadas. Tanto o motor como o gerador recorrem extensivamente a estas classes para desempenhar as suas funções. Algumas destas classes são recorrentes e usadas mais que uma vez em diferentes contextos. Assim, antes de introduzir como funciona o motor e o gerador é importante primeiro perceber quais são as classes que ambos usam, bem como a função de cada uma.

O objectivo de construção destas classes foi simplificar a construção do motor e gerador apenas. Nesta primeira etapa, foram deixadas para segundo plano questões como a modularidade e encapsulamento de dados, embora seja algo a considerar em etapas posteriores do trabalho.


\section{CoordsPolares}
\label{p1:coordsPolares}
Esta classe pretende abstrair os cálculos relacionados com coordenadas polares. A ideia é poderem ser criadas instâncias desta classe indicando os parâmetros correspondentes a coordenadas polares e as correspondentes coordenadas cartesianas poderem ser obtidas facilmente a partir da instância.




Designando o ponto central como C(cx,cy,cz) e um ponto P(px,py,pz) que se pretende localizar, temos que as coordenadas px e py podem ser obtidas da seguinte forma:\\

	$px = cx + r \times sin(\alpha)$
	
	$py = cy$
	
	$pz = cz + r \times cos(\alpha)$\\

A classe \textit{CoordsPolares} representa apenas uma forma fácil de trabalhar com este sistema de coordenadas. Esta classe possui variáveis de instância relacionadas com as suas coordenadas polares e cartesianas:

\begin{Verbatim}
	//Centro a partir do qual se 
	//considera as coordenadas polares
	Ponto3D centro;
	//Coordenadas polares
	float raio, azimuth;
	//Coordenadas rectangulares correspondentes 
	//às coordenadas polares
	Ponto3D cCartesianas;
\end{Verbatim}

O ponto chave desta classe é que as variáveis das coordenadas polares e cartesianas referem-se sempre ao mesmo ponto. Sempre que um dos parâmetros é alterado (através de uma das funções disponibilizadas), todos os restantes são actualizados em conformidade.

A classe tem um único construtor:

\begin{Verbatim}
CoordsPolares(Ponto3D c, float r, float az);
\end{Verbatim}

Neste construtor são pedidos os parâmetros referentes às coordenadas polares. O primeiro parâmetro \textit{c} corresponde ao ponto central, o segundo parâmetro r corresponde ao raio e o último parâmetro corresponde ao ângulo azimuth ($\alpha$). Ao receber estas coordenadas polares, o construtor seguidamente atualiza as variáveis de instância em conformidade, nomeadamente coloca na variável de instância \textit{cCartesianas} as coordenadas cartesianas correspondentes às coordenadas polares passadas como argumento. Essa actualização é feita pela função \textit{refreshCartesianas()}:

\begin{Verbatim}
void refreshCartesianas(){
	cRectangulares.x = centro.x + raio * sin(azimuth);
	cRectangulares.y = centro.y;
	cRectangulares.z = centro.z + raio * cos(azimuth);
}
\end{Verbatim}

Como se pode ver, esta função implementa apenas as fórmulas apresentadas anteriormente. Esta função é privada à classe, por isso não pode ser chamada em qualquer parte do código. É a própria classe que se responsabiliza por chamar esta função sempre que necessário.

Assim, dada uma instância desta classe é sempre possível saber as coordenadas cartesianas correspondentes através da função \textit{toCartesianas()}:

\begin{Verbatim}
Ponto3D toCartesianas() {
	return cCartesianas;
}
\end{Verbatim}


\section{CoordsEsfericas}
\label{p1:cEsfericas}
Esta classe pretende abstrair os cálculos relacionados com coordenadas esféricas. A ideia é poderem ser criadas instâncias desta classe indicando os parâmetros correspondentes a coordenadas esféricas e as correspondentes coordenadas cartesianas poderem ser obtidas facilmente a partir da instância. É também possível fazer o inverso, ou seja, indicar um ponto com coordenadas cartesianas e obter as respectivas coordenadas esféricas.

As coordenadas esféricas são constituídas por um raio $r$, por um ângulo $\theta$ (também designado por ângulo azimuth) e um ângulo $\phi$ (também designado por ângulo polar), conforme apresentado na figura \ref{p1:fig:p1_sphericalCoords}:



É possível saber as coordenadas cartesianas de um ponto P(px,py,pz) a partir das suas coordenadas esféricas através das seguintes fórmulas:\\

		$px = r \times cos(\theta) \times sin(\phi)$
		
		$py = r \times sin(\theta) \times sin(\phi) $
		
		$pz = r \times cos(\phi)$\\


Por outro lado, consegue-se saber também as coordenas esféricas de um ponto a partir das suas coordenadas cartesianas de acordo com as seguintes fórmulas:\\

		$r = \sqrt{px^2 + py^2 + pz^2}$
		
		$\theta = \arctan(py \backslash px) $
		
		$\phi = \arccos (pz \backslash r)$\\

De referir que ao contrário da classe \textit{CoordsPolares}, nesta classe optou-se por não se considerar um centro. Assume-se o centro como sendo (0.0,0.0,0.0) sempre. Considerou-se esta simplificação razoável na medida em que responde aos requisitos do motor e gerador nesta primeira fase.

A classe \textit{CoordsEsfericas} tem as seguintes variáveis de instância:

\begin{Verbatim}
// Coordenadas Esfericas
float raio, azimuth_ang, polar_ang;
// Coordenadas cartesianas
Ponto3D cCartesianas;
\end{Verbatim}

O ponto chave desta classe é que as variáveis das coordenadas esféricas e cartesianas referem-se sempre ao mesmo ponto. Sempre que um dos parâmetros é alterado (através de uma das funções disponibilizadas), todos os restantes são atualizados em conformidade.

Uma instância da classe \textit{CoordsEsfericas} pode ser criada indicando os parâmetros das coordenadas esféricas (para se saber as suas coordenadas cartesianas), ou indicando um ponto em coordenadas cartesianas (do qual se pretende saber as coordenadas esféricas), através dos seguintes construtores:

\begin{Verbatim}
CoordsEsfericas(float r, float az, float polar);
CoordsEsfericas(Ponto3D pto);
\end{Verbatim}

Caso a instância seja criada a partir de coordenadas polares, o construtor calcula as coordenadas cartesianas correspondentes através da função \textit{refreshCartesianas()}:

\begin{Verbatim}
void refreshCartesianas() {
	cCartesianas.z = raio * sin(polar_ang) * cos(azimuth_ang);
	cCartesianas.x = raio * sin(polar_ang) * sin(azimuth_ang);
	cCartesianas.y = raio * cos(polar_ang);
}
\end{Verbatim}

Caso a instância seja criada a partir de coordenadas cartesianas, o construtor calcula as coordenadas polares correspondentes através da função \textit{refreshEsfericas()}:

\begin{Verbatim}
void refreshEsfericas() {
	raio = sqrt(pow(cCartesianas.x, 2) + 
		pow(cCartesianas.y, 2) + 
		pow(cCartesianas.z, 2));
	polar_ang = acos(cCartesianas.y / raio);
	azimuth_ang = atan2(cCartesianas.x, cCartesianas.z);
}
\end{Verbatim}

Estas duas funções correspondem à implementação das fórmulas apresentadas anteriormente e garantem que as variáveis de instância correspondentes às coordenadas esféricas e cartesianas se referem sempre ao mesmo ponto. Ambas são funções privadas, pelo que é a própria classe que tem a responsabilidade de as chamar sempre que é necessário atualizar valores.

A qualquer momento, é possível saber as coordenadas cartesianas de uma instância pela função \textit{toCartesianas()}

\begin{Verbatim}
Ponto3D toCartesianas() {
	return cCartesianas;
}
\end{Verbatim}

Além destas funções, a classe possui ainda funções adicionais que permitem mudar a posição do ponto representado por cada instância. Sempre que uma destas funções é chamada tanto as variáveis de instância das coordenadas polares e cartesianas são atualizadas quer pela função refreshEsfericas() ou refreshCartesianas(). Estas funções que permitem mudar a localização do ponto, garantem ainda que $0 \leq \phi \leq \pi$ e que $0 \leq \theta \leq 2 \times \pi$

\newpage

\section{Figura}

Esta classe representa uma figura que será desenhada pelo motor. Representa por isso apenas um conjunto de pontos numa determinada ordem que correspondem a triângulos, que por sua vez formam uma figura.

Por representar um conjunto de pontos, sem surpresa, a sua única variável de instância é um vector de pontos:

\begin{Verbatim}
std::vector<Ponto3D> pontos;
\end{Verbatim}

A utilidade desta classe revela-se pelas funções que disponibiliza. Em primeiro lugar, disponibiliza um conjunto de funções que quando chamadas colocam no vector \textit{pontos} os pontos necessários para o desenho de uma figura em concreto. Dessa forma, estas funções permitem criar planos, caixas, círculos, cilindros e esferas. Estas funções serão explicadas em mais detalhe quando for apresentado o gerador onde será mostrado de que forma estas funções podem ser chamadas bem como de que forma geram os pontos.

Além das funções que permitem criar figuras destaca-se ainda a função \textit{toFicheiro()}, que permite guardar os pontos da figura num ficheiro cujo nome é passado como argumento.

\begin{Verbatim}
void toFicheiro(std::string filePath)
\end{Verbatim}

É ainda possível obter os pontos da figura pela função \textit{getPontos()}:

\begin{Verbatim}
std::vector<Ponto3D> getPontos()
\end{Verbatim}

\section{TinyXML-2}

Esta biblioteca foi usada para auxílio à leitura de ficheiros XML por parte do motor e pode ser encontrada no endereço: http://www.grinninglizard.com/tinyxml2/

\chapter{Conclusão}

O trabalho apresentado cumpre todos os requisitos propostos.
Foi feito um gerador de figuras com capacidade de gerar pontos para um plano, caixa, cone e esfera como pedido. Além destas figuras foram ainda desenvolvidas funções adicionais que permitem criar planos sem ser no eixo XZ bem como foram desenvolvidas funções para a criação de círculos e cilindros, algo que não era expressamente pedido.

No lado do motor, a aplicação consegue ler ficheiros XML e .3D e a partir dai desenhar as figuras com os pontos especificados nos ficheiros. Adicionalmente ao sugerido, incluiu-se também uma câmara colocada sobre uma esfera que permite ao utilizador ver a figura desenhada de vários ângulos. Incluiu-se ainda um pequeno menu para alterar o modo de visualização da figura.

O código produzido recorreu ao uso de classes, o que evitou bastantes repetições de código e sobretudo gerou um código fácil de ler, perceber e manter. Por estes motivos, considera-se como bastante sólido o trabalho desenvolvido.

Não obstante, existem aspectos em que o trabalho que poderiam ser melhorados e serão alvo de atenção no futuro. 

Em primeiro lugar, destaca-se a questão da câmara. Nesta fase implementou-se a câmara usando coordenadas esféricas. Decidiu-se que a camâra seria por isso apenas uma instância da classe \textit{CoordsEsfericas}. Deste modo mover a câmara corresponde apenas a chamar as funções definidas na classe. Este aspecto facilitou imenso a implementação da câmara, no entanto trouxe também algumas desvantagens. Sendo a câmara uma instância de \textit{CoordsEsfericas} significa que a câmara só pode ter coordenadas esféricas. Isto dificulta a adição de funcionalidades extra à câmara. Além disso, enquanto que é perfeitamente válido que as coordenadas esféricas possam referir um ponto ``no polo norte'' da esfera, tal não é verdade para a câmara, pois nesse caso o objeto pode deixar de se tornar visível. Isto deixa a entender que no futuro a câmara terá que pertencer a uma classe própria e muito provavelmente será esse o caminho a seguir.

Em segundo lugar, refere-se a modularidade e encapsulamento de dados, aspectos que foram deixados um pouco para segundo plano. A prioridade desde cedo foi ter código simples, funcional e fácil de ler. Tal implicou que muitas vezes quando confrontados com a decisão de manter algumas variáveis como públicas ou privadas a decisão tenha sido manter públicas. Exemplos disso são as classes Ponto3D (note-se a falta de getters e setters) e as classes das coordenadas esféricas e polares. Enquanto que ter variáveis públicas em classes como a Ponto3D seja relativamente irrelevante, tal já não é verdade para as classes das coordenadas. Como o objetivo principal do projeto não é ter bons módulos de dados nem bom encapsulamento dos mesmos, estes aspectos foram deixados para segundo plano, no entanto serão alvo de uma atenção mais cuidada no futuro.

\begin{appendices}
	\chapter{Código implementado}
	\section{PROLOG}
	
\begin{verbatim}



%--------------------------------- - - - - - - - - - -  -  -  -  -   -
% SIST. REPR. CONHECIMENTO E RACIOCINIO - MiEI/3

%--------------------Exercício 2 ------------- - - - - - - - - - -  -  -  -  -   -
% Base de Conhecimento do registo de eventos numa instituição de saúde

%--------------------------------- - - - - - - - - - - - - - - -
% SICStus PROLOG: Declaracoes iniciais

:- op( 900,xfy,'::' ).
:- set_prolog_flag( discontiguous_warnings,off ).
:- set_prolog_flag( single_var_warnings,off ).
:- set_prolog_flag( unknown,fail ).

/* permitir adicionar a base de conhecimento	*/

:-dynamic utente/4.
:-dynamic servico/4.
:-dynamic consulta/4.

% Extensao do predicado utente(IdUten,Nome,Idade,Morada) ->{V,F,D}

utente(1,gil,12,rua_Braga).

utente(20,gil,12,rua_Braga).

utente(2,carlos,20,rua_Guimaraes).
utente(3,sandro,30,rua_Lisboa).
utente(4,ana,10,rua_Varzim).
utente(5,filipa,15,rua_Coimbra).
utente(6,antonio,13,rua_Pacos).
utente(7,filipe,13,rua_Guimaraes).

-utente(ID,N,I,M):- nao(utente(ID,N,I,M)),
nao(excecao(utente(ID,N,I,M))).

-utente(13,joaquim,60,rua_Fafe).

% Nao sabemos a morada nem nunca vamos deixar saber que se saiba

utente(8,johnny,10,morada_desconhecido).

excecao(utente(ID,NO,I,R)):- utente(ID,NO,I,morada_desconhecido).

nulo(morada_desconhecido).


+utente(ID,NO,I,R) :: 
(solucoes( (ID,NO,I,R),(utente(ID,johnny,I,R),nao(nulo(R))),S ),
comprimento( S,N ), N == 0 
).

% Nao sabemos se mora em guimaraes ou em fafe							  

excecao(utente(9,carlos,60,guimaraes)).
excecao(utente(9,carlos,60,fafe)).								  


% Nao sabemos a idade ao certo do utente 

excecao(utente(10,lourenco,I,fafe)):- (I>=30,I=<40).

utente(11,filipe,30,morada_desconhecido).
utente(12,celia,22,morada_desconhecido).									 


%--------------------------------- - - - - - - - - - -  -  -  -  -   -
% Extensao do predicado servico(ID,Descricao,Instituicao,Cidade) ->{V,F,D}


servico(1,cardiologia,hospital_Braga,braga).
servico(2,urologia,hospital_Braga,braga).
servico(3,neurologia,hospital_Guimaraes,guimaraes). 
servico(4,cirugia_Geral,hospital_Guimaraes,guimaraes).
servico(5,ortopedia,hospital_Coimbra,coimbra).
servico(6,psiquiatria,hospital_Fafe,fafe).
servico(7,ginecologia,hospital_Fafe,fafe).
servico(10,cardiologia,hospital_Lisboa,lisboa).


-servico(ID,D,I,C):- nao(servico(ID,D,I,C)),
nao(excecao(servico(ID,D,I,C))).


% Nao sabemos se endocrinologia é prestado no hospital_Fafe ou no de hospital_Fafe_Novo				

excecao(servico(8,endocrinologia,hospital_Fafe,fafe)).
excecao(servico(8,endocrinologia,hospital_Fafe_Novo	,fafe)).


%--------------------------------- - - - - - - - - - -  -  -  -  -   -
% Extensao do predicado consulta(Data,IDUtente,IDServico,Custo) ->{V,F,D}
% data (ano,mes,dia)


consulta(data(2010,1,10),1,1,200).
consulta(data(2011,2,11),2,2,20).
consulta(data(2012,3,12),3,3,210).
consulta(data(2013,4,13),4,4,220).
consulta(data(2014,5,14),5,5,24).
consulta(data(2015,6,15),6,6,26).


-consulta(D,IDU,IDS,C):- nao(consulta(D,IDU,IDS,C)),
nao(excecao(consulta(D,IDU,IDS,C))).


% Nunca podemos saber a data em que o utente realizou uma consulta de um determinado servico

consulta(data_desconhecida,7,1,200). 
nulo(data_desconhecida).

excecao(consulta(D,IDU,IDS,C)):-consulta(data_desconhecida,IDU,IDS,C).

+consulta(D,IDU,IDS,C) :: 
(solucoes( (D,IDU,IDS,C),(consulta(D,7,1,C),nao(nulo(D))),S ),
comprimento( S,N ), N == 0 
).


% Nao sabemos quanto ele pagou por um determinado servico

excecao(consulta(D,IDU,IDS,C)):- 
consulta(D,IDU,IDS,custo_desconhecido).

consulta(data(2014,6,15),1,2,custo_desconhecido).

% Nao sabemos ao certo quanto ele pegou por um determinado servico
excecao(consulta(data(2015,6,15),1,3,C)):- (C>=10,C=<100).


% Nunca podemos vir a saber que tipo de servico o utente teve naquela data 

consulta(data(2015,6,15),1,servico_desconhecido,20).
nulo(servico_desconhecido).
excecao(consulta(D,ID,IDS,C)):- consulta(D,ID,servico_desconhecido,C).

+consulta(D,ID,IDS,C) :: 
(solucoes((D,ID,IDS,C),(consulta(data(2015,6,15),1,IDS,C),
nao(nulo(IDS))),S),
comprimento(S,N),
N==0).

%--------------------------------- - - - - - - - - - -  -  -  -  -   -
% Extensão do predicado que permite a insercao de conhecimento: Termo -> {v, F}

inserirUtente(ID,NO,I,M):-evolucao(utente(ID,NO,I,M)).
inserirServico(ID,D,I,C):-evolucao(servico(ID,D,I,C)).
inserirConsulta(D,IDU,IDS,C):-evolucao(consulta(D,IDU,IDS,C)).


%--------------------------------- - - - - - - - - - -  -  -  -  -   -
% Extensão do predicado que permite a remocao de conhecimento: Termo -> {v, F}

removerUtente(ID,NO,I,M):-remover(utente(ID,NO,I,M)).
removerServico(ID,NO,I,C):-remover(servico(ID,NO,I,C)).
removerConsulta(D,IDU,IDS,C):-remover(consulta(D,IDU,IDS,C)).


% Contar o numero de utentes

conta_Utente(Numero):-
(findall((ID,N,I,M),(utente(ID,N,I,M)),S),comprimento(S,Tamanho), 
Numero is Tamanho).


% Contar o numero de Servicos

conta_Servico(Numero):-
(findall((ID,N,I,M),servico(ID,N,I,M),S),comprimento(S,Tamanho), 
Numero is Tamanho).

% Contar o numero de Consultas

conta_Consulta(Numero):-
(findall((ID,N,I,M),consulta(ID,N,I,M),S),comprimento(S,Tamanho),
Numero is Tamanho).



%--------------------------------- - - - - - - - - - -  -  -  -  -   -
% Extensão do predicado que permite a remoção de conhecimento: Termo -> {v, F}

remover(Termo):-
solucoes(Inv,-Termo::Inv,LInv),
remocao(Termo),
teste(LInv).

remocao(Termo):-
retract(Termo).
remocao(Termo):-
assert(Termo),!,fail.
%--------------------------------- - - - - - - - - - -  -  -  -  -   -


evolucao( Termo ) :-
solucoes( Invariante,+Termo::Invariante,Lista ),
insercao( Termo ),
teste( Lista ).

insercao( Termo ) :-
assert( Termo ).
insercao( Termo ) :-
retract( Termo ),!,fail.

teste( [] ).
teste( [R|LR] ) :-
R,
teste( LR ).

%--------------------------------- - - - - - - - - - -  -  -  -  -   -
% Extensao do meta-predicado demo: Questao,Resposta -> {V,F}

demo( Questao,verdadeiro ) :-
Questao.
demo( Questao, falso ) :-
-Questao.
demo( Questao,desconhecido ) :-
nao( Questao ),
nao( -Questao ).

%--------------------------------- - - - - - - - - - -  -  -  -  -   -
% Extensao do meta-predicado nao: Questao -> {V,F}

nao( Questao ) :-
Questao, !, fail.
nao( Questao ).

%--------------------------------- - - - - - - - - - -  -  -  -  -   -

solucoes( X,Y,Z ) :-
findall( X,Y,Z ).

comprimento( S,N ) :-
length( S,N ).


/* ==================Invariante========================*/


% Nao deixa inserir o mesmo conhecimento em relacao aos utentes

+utente(ID,NO,I,M) :: 
(solucoes((ID,NO,I,M),utente(ID,NO,I,M),S),comprimento(S,N), N==1).

% Nao deixa inserir o mesmo id em relacao aos utentes

+utente(ID,_,_,_) :: 
(solucoes(NO,utente(ID,NO,_,_),S),comprimento(S,N), N==1).


/*=======Servicos ==============*/ 

% Nao deixa inserir o mesmo conhecimento em relacao aos servicos

+servico(ID,D,I,C) :: 
(solucoes((ID,D,I,C),servico(ID,D,I,C),S),comprimento(S,N),N==1).


% Nao deixa inserir servicos com o mesmo ID

+servico(ID,_,_,_) :: 
(solucoes(D,servico(ID,D,_,_),S),comprimento(S,N),N==1).

/*=======Consultas ==============*/ 

% Nao deixa inserir o mesmo conhecimento em relacao as consultas

+consulta(D,IDU,IDS,C) ::
(solucoes((D,IDU,IDS,C),consulta(D,IDU,IDS,C),S),comprimento(S,N),N==1).

% So podemos adicionar consultar se o id do utente existir 

+consulta(_,ID,_,_) ::
(solucoes(NO,utente(ID,NO,_,_),S),comprimento(S,N),N==1).

% So podemos adicionar consultar se o id do servico existir


+consulta(_,_,IDS,_) :: 
(solucoes(NO,servico(IDS,NO,_,_),S),comprimento(S,N),N==1).


/*===========Remover===================*/

% Nao deixar remover utentes que estejam nas consultas

-utente(ID,_,_,_) :: (nao(consulta(_,ID,S,_)),nao(utente(ID,_,_,_))).

/*=========Servicos =============*/
% Nao deixar remover um servico que esta nas consultas 

-servico(ID,X,Y,Z) :: (nao(consulta(P,L,ID,K)),nao(servico(ID,X,Y,Z))).

\end{verbatim}

\section{Java}
\subsection{menu}
\begin{verbatim}
public class Menu {

public Menu(){
int i=0;
System.out.println("*****************************************************************");
System.out.println("***************************SRCR**********************************");
System.out.println("************************Exercício 2******************************");
System.out.println("*****************************************************************");
System.out.println("\n\n");
String args[]={ ". Número de utentes resgistados.",
". Número de Servicos Registados.",
". Número de consultas.",
". Encontrar informações sobre um utente com um dado nome.",
". Informações sobre um determinado Serviço.",
". Pesquisar consultas por data.",
". Para um data e um id ver os servicos que ele foi e o custo.",
". Dado um id do utente saber quem é esse utente.",
". Dado um id do serviço saber qual é esse servico.",
". Ver as consultas que nao se sabe a data.",
". Ver as consultas que nao se sabe o custo.",
". Ver os utentes em que a morada seja desconhecida.",
". Adicionar utente",
". Adicionar servico",
". Adicionar consulta",
". Remover utente",
". Remover servico",
". Remover consulta",
". Demonstração do utente",
". Demonstração do servico",
". Demonstração da consulta",
". Sair."
};

while(i<args.length-1) {

System.out.println(i+1+args[i]);

i++;
}

System.out.println(0+args[i]);
System.out.print("\n");

	}
}
\end{verbatim}




\end{appendices}

\end {document}