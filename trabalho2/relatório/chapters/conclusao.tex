\chapter{Conclusão}

A utilização de Programação Lógica Estendida surgiu como o principal desafio neste exercício. Com este exercício conseguimos perceber melhor o conceito de conhecimento imperfeito bem como o significado de cada um dos valores nulos que tínhamos estudado, bem como a utilidade para situações da vida real.

A aplicação dos três tipos de conhecimento imperfeito não levantou problemas e, todos os resultados dos testes e exemplos práticos foram os esperados. Após identificadas as condições de inserção de registo, a construção dos invariantes foi também bastante simples e rápida. 

 Contudo tivemos algumas dificuldades na criação de exemplos relacionados com a realidade para a criação da base de conhecimento. Analisando os resultados obtidos, temos que todas as funcionalidades funcionam de acordo com as nossas expetativas e estão de acordo com a nossa base de conhecimento. O facto de ser implementada uma interface em JAVA com recurso à biblioteca JASPER, fez com que a interação com o utilizador ficasse mais simplificada. 



