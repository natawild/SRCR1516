\begin{abstract}

O presente relatório documenta o segundo trabalho prático da Unidade Curricular de Sistemas de Representação Conhecimento e Racocínio.
Nesta fase o objetivo será construir um mecanismo de representação de conhecimento com capacidade para caracterizar um universo de discurso na área da prestação de cuidados de saúde. O objetivo é demonstrar as funcionalidades subjacentes à programação em lógica estendida e à representação de conhecimento imperfeito,
recorrendo à temática dos valores nulos. Em termos gerais, foi usada a linguagem PROLOG, esta que utiliza um conjunto de fatos, predicados e regras de derivação de lógica e haverá ainda uma interação com o sistema criado para a implementação do caso prático deverá ser desenvolvida em JAVA, com recurso à biblioteca JASPER.
Neste relatório pretende-se apresentar a forma como a aplicação foi construída bem como explicar algumas decisões tomadas.


\end{abstract}