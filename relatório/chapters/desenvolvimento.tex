\chapter{Introdução}
\label{cap:p1}
Este primeiro capítulo tem como objetivo apresentar uma breve introdução ao exercício a realizar. Sendo assim, é necessário perceber os motivos que levaram à resolução do exercício assim como os objetivos pretendidos.
A Programação em Lógica é um tipo específico de programação cujo objetivo é a implementação de um programa cujo conteúdo se prende em factos (registos que se sabem verdadeiros), predicados (associados aos factos) e regras. A este programa podem ser estruturadas questões sobre o seu conteúdo e obter-se-ão respostas válidas e corroboradas pela lógica em si.
A programação em lógica baseia-se em dois princípios básicos para a “descoberta” das respostas (soluções) a essas questões: lógica, usada para representar os conhecimentos e informação, e Inferência, regras aplicadas à Lógica para manipular o conhecimento.




\section{Motivação e Objetivos}
\label{p1:MotivObj}
O PROLOG é uma linguagem declarativa, pois fornece uma descrição do problema que se pretende computar utilizando uma coleção de factos e regras lógicas que indicam como deve ser resolvido o problema proposto. Sendo também uma linguagem que é especialmente associada com a inteligência artificial e linguística computacional foi um dos grandes motivos que nos levou a querer aprender este tipo de linguagem mais direcionado ao conhecimento do que aos algoritmos. 


Após os conhecimentos adquiridos na linguagem de programação lógica PROLOG, este exercício surge com o objetivo de consolidar conhecimentos e obter experiência e prática face a problemas de programação em lógica. O objetivo final será a construção de um programa capaz de armazenar conhecimento sobre registo de eventos numa instituição de saúde e através deste solucionar questões deste tema.



\section{Estrutura do documento}
\label{p1:Estrutura}
O presente relatório encontra-se organizado em X capítulos. Sendo que neste primeiro introduzimos a linguagem e o tema a tratar menciona-se também a motivação e os objetivos que nos levaram à realização deste exercício. 
No segundo capítulo será feito um estudo prévio da linguagem de modo a que o leitor possa entender o exercício. No terceiro capítulo explicaremos o que foi desenvolvido para a implementação do exercício. No quarto capítulo apresentaremos as conclusões e interpretação dos resultados obtidos. Por fim nos últimos dois capítulos será apresentada a bibliografia consultada, no quinto capítulo e os anexos no sexto. 




\chapter{Preliminares}
\label{cap:p2}
Neste capítulo vao ser apresentados alguns conceitos fundamentais para a elaboraçãoo deste
exercício e algumas das ferramentas fundamentais para a elaboração do mesmo.


\section{Estudos anteriores}
\label{p2:estudp}

Para a realização deste trabalho foram necessários alguns conhecimentos anteriores sobre programação em lógica e, posteriormente, o uso da linguagem de programação PROLOG.
Este conhecimento foi adquirido ao longo das aulas teóricas ( programação em lógica) e aulas
teórico-práticas (PROLOG) de Sistemas de Representação de Conhecimento e Raciocínio.
Sobre estes conhecimentos, devemos destacar todos os conceitos que foram aprendidos tais
como o que são predicados, o que são cláusulas, o que é a base de conhecimento, entre outros
conceitos de programação em lógica que serão explicados ao longo deste documento.
Após termos alguns conhecimentos de programação em lógica falta colocá-los em prática,
e, é aqui, que entram os conhecimentos de PROLOG e da ferramenta \textit{SICSTus} usada para
compilar e interpretar o código desenvolvido nesta linguagem.
\\


Para o desenvolvimento desses predicados foi necessário fazer uma análise dos conhecimentos de cada um dos membros sobre o tema e acompanhado de uma pequena pesquisa sobre as características destes.

O estudo inicial passou por caracterizar um utente, isto é, definir todos os parâmetros que definem o que é ser um utente. A conclusão a que se chegou foi que um utente é definido por nome, serviço a que está a ser inscrito ou consultado, profissional atribuído ou responsável por esse serviço e a instituição onde o profissional labora e onde o utente está a ser atendido, que coincidem obviamente. Na realidade, o utente é composto por muitos outros factores, tal como, número de utente, número de contribuinte, número de cartão de cidadão entre muitos outros mas que para a resolução dos critérios designados como mínimos apenas são requeridos aqueles mencionados anteriormente, pois as questões não pretendem incidir nesses campos.
\\

Um serviço é caracterizado por designação, isto é, o nome do serviço, que terá de ser elucidativo, por exemplo, “cirurgia” e por instituição, que corresponde ao nome do local onde se presta esse serviço aos utentes, e como tal será algo como “hospital....” ou “centro de saúde...” ou algo semelhante que reflita que esse local é um local em que se prestam determinados serviços na área da saúde. Na realidade serviços é muito mais complexo e passa por ter chefes de equipas, e por vezes especificação detalhada, como por exemplo, “cirurgia cardíaca” que é diferente de “cirurgia cerebral”. Ou seja dentro do serviço existem categorias e para cada categoria existem chefes de equipas e as respectivas equipas. Generalizando o exemplo, teríamos um chefe do serviço geral “cirurgia” que teria uma equipa de chefes de categorias desse serviço e cada categoria teria uma equipa de profissionais. Mas nada disto é utilizado para cumprir os requisitos mínimos e tal como nos casos anteriores é sim algo que pode ser utilizado como uma extensão do projeto.  
\\
Um profissional é caracterizado pelo seu nome, serviço em que está inserido, sendo que pode estar em mais que um serviço e a instituição onde labora. E estes são apenas os requisitos mínimos que permitem ao sistema determinar as respostas a todas as questões, mas um profissional é algo mais que isto. Um profissional tem também um chefe associado, tem uma especialidade, tem um identificador dentro da instituição, visto que um profissional pode exercer funções em mais que uma instituição, e em mais que um serviço. Para além destes possui também informações semelhantes ás do utente que dizem respeito ao facto de serem cidadãos. E que são características possíveis de utilizar em futuras extensões. 
\\

Uma instituição é caracterizada apenas pelo seu nome por forma a simplificar este sistema visto que os requisitos mínimos não pretendem questionar nada de especial que implique que a instituição tenha de ter algo mais no seu predicado além do seu nome. Mas como é óbvio uma instituição  é muito mais que um nome, isto é, quando alguém pensa numa instituição não pensa unicamente no seu nome. Na realidade uma instituição é composta por vários departamentos, um dos quais é o departamento responsável pela aplicação dos diversos serviços e é nesse que se encontram os profissionais, possui colaboradores, outros serviços, como cafetarias, salas de espera, balcões de atendimento ou de reclamações, entre muitos outros. No entanto, restringe-se esse conceito ao simples facto de ser um local onde profissionais prestam serviços a utentes.


\section{Programação em Lógica e PROLOG}
\label{p2:proglogprolog}
De modo a que a leitura deste documento seja perceptível em termos de conceitos e símbolos é necessário fazer referências breves a noções básicas de PROLOG, a linguagem em que é desenvolvido este trabalho.
Tal como foi mencionado anteriormente uma linguagem de programação lógica utiliza a lógica para representar conhecimento e inferências para manipular informação. Um programa neste tipo de programação possui então os seguintes parâmetros:

\begin{itemize}
	\item Factos - constatações sobre algo que se conhece e se sabe verdadeiro, por exemplo cor( azul )
	\item Predicados – implementam relações, por exemplo o predicado filho( filho, pai ) implementa a relação de descendência direta (ser filho de)
	\item Regras – utilizadas para definir novos predicados. 
\end{itemize}

Estes são alguns exemplos dos conhecimentos base para perceber a programação em lógica.
Após se ter estes conhecimentos, é necessário traduzir estes e aplicá-los na linguagem PROLOG.
Deixamos, então, alguns exemplos importantes para a usar:

\begin{itemize}
	\item .  utilizado para terminar uma declaração;
	\item :-  significa “se”;
	\item ,  possui o significado “e”;
	\item ;  significa “ou”;
	\item //  representa a unificação;
\end{itemize}

É ainda necessário referir que as variáveis representam-se por maiúsculas e constantes, predicados e factos com minúsculas.
Com estas noções como base passa-se agora ao desenvolvimento das tarefas do exercício.



\chapter{Descrição do Trabalho e Análise de Resultados}
\label{cap:p3}

Nesta parte do documento serão explicitadas todas as etapas de resolução dos desafios fornecidos bem como todas as decisões efetuadas no processo.


\section{Base de Conhecimento}
\label{p3:baseConhe}

A base de Conhecimento define bases de dados ou conhecimento acumulados sobre determinado assunto.
Para a elaboração deste exercício tornou-se importante definir uma base de conhecimento
que possa responder aos pedidos do enunciado.

\subsection{Instituições Existentes }

\begin{verbatim}
instituição(nome). 
instituicao( hospitalGuimaraes ). 
instituicao( hospitalBraga ).
instituicao( hospitalBarcelos ).
\end{verbatim}

\subsection{Serviços Existentes}

\begin{Verbatim}
servico(nome).

servico( cardiologia ). 
servico( cirugiageral ).
servico(  neurologia ).
\end{Verbatim}

\subsection{Profissionais}
\begin{Verbatim}
profissional(codigo, nome).

profissional(1,marcus).
profissional(2,maria).
profissional(3,jorge).
\end{Verbatim}

\subsection{Utentes }

\begin{Verbatim}
utente(codigo,nome).

utente(1,jose).  
utente(2,carlos). 
utente(3,maria). 
\end{Verbatim}

