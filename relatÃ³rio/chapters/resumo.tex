\begin{abstract}

O presente relatório documenta o primeiro trabalho prático da Unidade Curricular de Sistemas de Representação Conhecimento e Racocínio.
Nesta primeira fase o objetivo foi construir um mecanismo de representação de conhecimento para o registo de eventos numa instituição de saúde. Em termos gerais, foi usada a linguagem PROLOG, esta que utiliza um conjunto de fatos, predicados e regras de derivação de lógica. Uma execução de um programa é, na verdade, uma prova de um teorema, iniciada por uma consulta. 
Neste relatório pretende-se apresentar a forma como a aplicação foi construída bem como explicar algumas decisões tomadas.




O relatório encontra-se organizado em 3 partes. Na primeira parte apresentam-se as classes usadas, explicando qual a função de cada uma e das suas respectivas funções. Na segunda parte apresenta-se o gerador. Indica-se quais as figuras para as quais é possível gerar pontos bem como de que forma os pontos de cada figura são gerados. Na última parte apresenta-se o motor, nomeadamente a forma como os ficheiros de pontos são lidos, como os pontos são desenhados, bem como algumas considerações sobre a câmara e menus da aplicação.


\end{abstract}