\begin{abstract}

O presente relatório documenta o primeiro trabalho prático da Unidade Curricular de Sistemas de Representação Conhecimento e Racocínio.
Nesta primeira fase o objetivo foi construir um mecanismo de representação de conhecimento para o registo de eventos numa instituição de saúde. Em termos gerais, foi usada a linguagem PROLOG, esta que utiliza um conjunto de fatos, predicados e regras de derivação de lógica. Uma execução de um programa é, na verdade, uma prova de um teorema, iniciada por uma consulta. 
Neste relatório pretende-se apresentar a forma como a aplicação foi construída bem como explicar algumas decisões tomadas.



\end{abstract}